
\chapter{Experiences}

EXPERIENCES WITH NEW LANGUAGES

NAME:         STELLA ANDRADE


GitHub and LaTeX

I was prepared about using LaTeX in this course, but the use of Github caught me off guard.
We were going to use Github as a team, to modify just one file with all of our modifications, and helping each other. The idea sounded awesome, but in practice it didn't came out well. Uploading the files, each on our own repository was easy, but trying to upload everything in a shared root wasn't an easy task. When we thought we were getting the hang of forking and pushing, the files didn't upload as we wanted, so instead we just gave the repository maker the modified files.

With LaTeX there were good and bad sides. In LaTeX you get this pretty elegant result in your documents, everything being in the place you wanted to and staying there, and all the specific titles and such had their own tag. But that last thing is also its downside. Having to search the specific LaTeX libraries just to get a certain-colored sidebar or a different image box, and the raw document that could let down anyone. I think it would have been easier or more fun being able to alter the style like it's done in HTML.


ANDROID

The first project. I remember when hearing 'Android application' it felt like it was a big thing.
We had various ideas for the project, but most of them were rejected. But in the end, the simple idea was chosen.
We started slow with the design. The design part was really fun, having to use XML to set the style of every object. But in the end, we left the most important thing for last.
We couldn't finish the project because in the end we couldn't get a mobile phone with Android to test our program. We could have tested in the Android emulator, but the features that we needed to use didn't work well in the emulator.


PYTHON

This was one of the languages I was wating to learn. Having all this powerful features in this little interpreter is really great.
I had heard that Python was good for making games, and having to implement a game for the course was a really great deal, at first.
The specifications for the game were pretty harsh, leaving us with little to none possibilities besides the sound novel, but we wanted to make something different. We found out this big PyGame community, where everyone shared their games, let them be in process or finished, and this webpage with books and sample code for practice. We used one of the sample codes in the books as a base for our game, but it wasn't allowed at the end.
I'm interested in making a little game using this language, without all this preassure of having it on time.


HASKELL

Well, this was new for all of us. Changing our minds to the 'R gear' as it said wasn't a very easy task. But looking at all the tutorials and books all I have to say is that they're very... friendly. I mean, children illustrations, funny explanations... they were all towards making the learning process somehow lighter. And it sorta worked.
Having to survive without fors or whiles, but with recurssion and heads and tails came to be somehow fun.
In the end, one of the hardest parts of this language was fully understanding the IO type and the pure/impure functions.



NAME:         DANNY PONCE


GITHUB

The experience of working with the GitHub repository was sincerily very pleasant, but it was hard to cope using this repository, starting from creating my CV. Also when creating the keys, I installed Github for Windows, but then I didn't know when it generated the keys, but they were created automatically. I didn't know about cloning nor uploading a file from the command line. After that, I learned managing my projects, cloning them in my laptop and uploading my files from the command line using the commit command. I also learned to work in groups using the fork option in Github, to share a project from another Github repository.


LATEX

My experience using LaTeX was not very of my liking because all of my reports and documents I've done them in Word. To use LaTeX I had to install MikTeX for Windows, and I downloaded various CV models for my first homework. Working in MikTeX was a lot of hard work for me, and that's why I didn't like it.


ANDROID

The experience of creating an Android applicarion was fantastic, even though I didn't get to finish the project because of not having a mobile device to test our application. I learned to use Eclipse to create an Android Application, and it wasn't very hard for me because Eclipse uses Java language. While making the graphic interface in the project it was very interesting how you program the specific function each object did. Also, I learned to use the GPS connection for Android mobiles, but the problem was when while it was being tested in the Android emulator it worked, but when testing it on a mobile phone it didn't work. That's why not having a mobile was a reason to not present the project. But the experience in working in Eclipse creating an Android application was very interesting


PYTHON

The experience of having learned the programming basics with Python was very good for me, and in my opinion Python is an excelent programming language. Python is multiparadigm, you can program using different programming styles. Python's sintaxis is sinthetic, and everything is where it's supposed to be, and everything behaves in a very logical way. Besides, being an interpreted language with indentation, you get used to write in a clear and elegant way.
What became hard for me in Python was the use of sound with the pygame library, because in my project there were many sounds and the ones I wanted didn't get played. At the end I found out where were my errors.


HASKELL

The experience of having to program in Haskell in the project was truly hard for me. It is a very different language from the ones we have been using. When we were generating random numbers it got very complicated. Trying to understand this language was very confusing, but in the end, we as a group we got to develop the project. I insist, from all of the languages, Haskell was the hardest for me.


NAME:         KEVIN SILVA


GitHub

This system was something new for me because I didn't knew about systems that did version control. It was a solution of a problem that always arises when making an application. But I had some problems installing it at start, and then when controlling its functions because the graphic interface didn't gave enough options, and what was needed to do was to use the command line, and that made it difficult for me. It also was uncomfortable having to do commits every now and then.
Overall, it was good that a tool like this was shown to us, they could help us in future projects in the next semesters.

LATEX

This new little language was easy for me. At the start it was something very new, and I didn't got used to it since we all had used Word, PowerPoint and the rest of the Office suite. But after researching a little more, I could find out the different commands that LaTeX gives us. One of the cons was that the graphic interface should be fixed, but learning this language will be very useful for me because nowdays it's used in a big scale in the professional field to make documents, letters, and presentations.


ANDROID

Android nowdays is a great platform to develop applications, that's why I think it was an excelent idea that we got to learn to use it. While developing our application I had some problems installing the platform, because some libraries and such were needed to install the needed plugin in Eclipse. Also installing the emulator, because we didn't have a device with Android in it to test the application. This was a big limitation, so we didn't get to show our application because in the emulator it was hard do emulate the GPS that oud application needed to work correctly. In the developing stage of our application was also a little complicated to learn XML, but on the other side, programming in Java wasn't a big deal for me. Other thing that was difficult to know was which Android version should we work without having the device in the first place. In conclusion, it was a little hard in the installing part, but the learning time wasn't long and the results were good.


PYTHON

This language was also complicated to learn. It introduced me to a new way to think, besides of not having programmed with an interpreted language before making me have a hard time with some punctuations in if sentences and such. But it came out to be an easy language to learn and that gives us many facilites when programming because, as being interpreted, we didn't had to call out the data types because Python knew them right away. One of the cons I'll give Python is the graphic interface, it was very poor for the big potential this language has.


HASKELL

Haskell was a language that was very difficult to learn, since it was functional, its programming style had to be different. You couldn't use loops, and everything had to be done recursively. The sintaxis was also different and a little hard to learn. One of the pros of this language is that, since everything is recursive, the functions came out to be really small comparing them with other imperative languages where sometimes the functions were really extense and even really hard to understand.
